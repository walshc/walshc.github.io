\documentclass[11pt]{beamer}
\usetheme{Boadilla}
\usecolortheme{dolphin}
\beamertemplatenavigationsymbolsempty
\usepackage{comment,graphicx,xcolor}
\definecolor{navy}{RGB}{0,0,125}
\hypersetup{colorlinks=true, linkcolor = navy, urlcolor = navy, citecolor = navy}
\setbeamertemplate{enumerate items}[default]

\title[Tutorial 1]{Tutorial 1}
\author{Statistics 2 for IBA}
\date{Tilburg University}

\begin{document}
\begin{frame}[plain]
  \titlepage
\end{frame}

\begin{frame}{Introduction}
  \begin{itemize}
    \item In this tutorial we will explore the \texttt{f150.sav} dataset, which featured in the Resit exam in January 2021.
    \item The dataset contains information from advertisements for second-hand Ford F-150 trucks on Craigslist, a US website similar to Marktplaats here in NL.
    \item The data contains the asking price, the year of manufacture, the mileage (odometer reading), the truck's color and an indicator for whether truck is in good condition or not.
  \end{itemize}
\end{frame}


\begin{frame}{Dataset Description as seen in TestVision}
  \includegraphics[width=\textwidth]{f150-exam-intro.png}
\end{frame}

\begin{frame}{Exercises}
  \begin{enumerate}
    \item Open the \texttt{f150.sav} data file and inspect the data. Look at \emph{Data View} and \emph{Variable View}.
    \item Create a histogram of the variable \emph{Price} using \texttt{Graphs}$\rightarrow$\texttt{Legacy Dialogs}$\rightarrow$\texttt{Histogram}.
    \item Obtain descriptive statistics (mean/min/max/SD) of the variable \emph{Price}.
    \item Create a scatter plot of \emph{Year} against \emph{Price}, with \emph{Year} on the horizontal axis and price on the vertical axis. Interpret it.
    \item Compute the covariance and correlation between \emph{Year} and \emph{Price}. Interpret them.
    \item Create the variable \emph{Age} from \emph{Year}. Keep in mind that the advertisements were shown in 2020.
    \item Compute the covariance and correlation between \emph{Age} and \emph{Price}. Relate this to what you found in Q5.
    \item Study the relationship between \emph{Age} and \emph{Year}. Create a scatter plot and compute the covariance and correlation. Interpret them.
  \end{enumerate}
\end{frame}

\begin{frame}{Bonus Questions: Q1 from the January 2021 Resit}
  \begin{center}
    \includegraphics[width=\textwidth]{f150-exam-q1.png}
  \end{center}
  You were asked to type a number into the box.
\end{frame}

\begin{frame}{Bonus Questions: Q2 from the January 2021 Resit}
  \begin{center}
    \includegraphics[width=\textwidth]{f150-exam-q2.png}
  \end{center}
  Multiple choice question.
\end{frame}

% \begin{comment}
\begin{frame}{Q1: Data View}
  \includegraphics[width=\textwidth]{f150-data-view.png}
\end{frame}

\begin{frame}{Q1: Variable View}
  \includegraphics[width=\textwidth]{f150-variable-view.png}
  \begin{itemize}
    \item Price, year and odometer are continuous numerical variables.
    \item Paint color is a \emph{string} (character) variables.
    \item Good condition is an indicator (dummy) variable: it equals 1 when the truck is in good condition and 0 when it is not.
  \end{itemize}
\end{frame}

\begin{frame}{Q2: Creating a Histogram}
  \begin{center}
    \includegraphics[width=0.7\textwidth]{f150-graphs-legacy-histogram.png}
  \end{center}
\end{frame}

\begin{frame}{Q2: Creating a Histogram}
  \begin{center}
    \includegraphics[width=0.7\textwidth]{f150-histogram-dialog.png}
  \end{center}
\end{frame}

\begin{frame}{Q2: Creating a Histogram}
  \begin{center}
    \includegraphics[width=0.7\textwidth]{f150-histogram-price.png}
  \end{center}
  \begin{itemize}
      \item Prices vary between (slightly above) zero and about 70k.
      \item Most prices are below 40k.
      \item The distribution is skewed to the right.
  \end{itemize}
\end{frame}

\begin{frame}{Q3: Obtaining Descriptive Statistics}
  \begin{center}
    \includegraphics[width=0.5\textwidth]{f150-analyze-descriptives.png} \\
    \vspace{1cm}
    \includegraphics[width=0.7\textwidth]{f150-descriptive-options.png}
  \end{center}
\end{frame}

\begin{frame}{Q3: Obtaining Descriptive Statistics}
  \begin{center}
    \includegraphics[width=0.7\textwidth]{f150-descriptives.png}
  \end{center}
  \begin{itemize}
    \item 500 observations in total.
    \item The lowest advertised price is \$499, the highest is \$71,900.
    \item The average is \$17,854.54.
    \item The standard deviation is \$12,575.61.
  \end{itemize}
\end{frame}

\begin{frame}{Q4: Creating a Scatter Plot}
  \begin{center}
    \includegraphics[width=0.7\textwidth]{f150-graphs-legacy-scatter.png}
  \end{center}
\end{frame}

\begin{frame}{Q4: Creating a Scatter Plot}
  Put variable on the horizontal axis in the $x$-axis box. \\
  Put variable on the vertical axis in the $y$-axis box.
  \begin{center}
    \includegraphics[width=0.5\textwidth]{f150-scatter-dialog.png}
  \end{center}
\end{frame}

\begin{frame}{Q4: Creating a Scatter Plot}
  \begin{center}
    \includegraphics[width=0.9\textwidth]{f150-scatter-year-price.png}
  \end{center}
  \begin{itemize}
    \item Positive relationship between year and price.
    \item Somewhat nonlinear relationship: cars from before 2000 have a low price, but starting 2000 onwards the relationship is linear.
  \end{itemize}
\end{frame}

\begin{frame}{Q5: Covariance and Correlation}
  \begin{center}
    \includegraphics[width=0.6\textwidth]{f150-analyze-correlate-bivariate.png}
  \end{center}
\end{frame}

\begin{frame}{Q5: Covariance and Correlation}
  Click \texttt{Options\ldots} to add covariances to the table.
  \begin{center}
    \includegraphics[width=\textwidth]{f150-correlate-options.png}
  \end{center}
\end{frame}

\begin{frame}{Q5: Covariance and Correlation}
  \begin{center}
    \includegraphics[width=0.5\textwidth]{f150-correlation-table-price-year.png}
  \end{center}
  \begin{itemize}
    \item Covariance is \$65,380.189. This indicates a positive relationship, but is otherwise not easily interpretable.
    \item Correlation is 0.720. This indicates a strong positive linear relationship (because the correlation can be at most 1).
  \end{itemize}
\end{frame}

\begin{frame}{Q6: Computing a New Variable, where $age=2020-year$}
  \begin{center}
    \includegraphics[width=0.6\textwidth]{f150-transform-compute-variable.png}
  \end{center}
\end{frame}

\begin{frame}{Q6: Computing a New Variable, where $age=2020-year$}
  For example, if year is 2019, age is 1. If year is 2018, age is 2.
  \begin{center}
    \includegraphics[width=0.7\textwidth]{f150-compute-age.png}
  \end{center}
\end{frame}

\begin{frame}{Q6: Computing a New Variable}
  A new column appears in the dataset. Check that the rows make sense! Remember that the advertisements were from 2020.
  \begin{center}
    \includegraphics[width=0.6\textwidth]{f150-new-age-column.png}
  \end{center}
\end{frame}

\begin{frame}{Q7: Covariance and Correlation with Price and Age}
  \begin{center}
    \includegraphics[width=0.4\textwidth]{f150-correlation-table-price-age.png}
  \end{center}
  \begin{itemize}
    \item Covariance is -\$65,380.189. This indicates a negative relationship, but is otherwise not easily interpretable. It has the same as with \emph{Year} instead of \emph{Age} apart from the negative sign.
    \item Correlation is -0.720. This indicates a strong negative linear relationship (because the correlation cannot be lower than $-1$). It has the same as with \emph{Year} instead of \emph{Age} apart from the negative sign.
  \end{itemize}
\end{frame}

\begin{frame}{Q8: Relationship Between Year and Age}
  \begin{center}
    \includegraphics[width=0.8\textwidth]{f150-scatter-year-age.png}
  \end{center}
  \begin{itemize}
      \item Perfect negative linear relationship!
      \item This is because $year=2020-age$ for every observation.
  \end{itemize}
\end{frame}

\begin{frame}{Q8: Relationship Between Year and Age}
  \begin{center}
    \includegraphics[width=0.4\textwidth]{f150-correlation-year-age.png}
  \end{center}
  \begin{itemize}
    \item Correlation exactly $-1$. This is a perfect negative linear relationship.
    \item Both year and age have the same variance (52.095). The covariance is the negative of the variance (-52.095).
  \end{itemize}
\end{frame}

\begin{frame}{Bonus Questions from the Exam}
  \textbf{Q1:}
  \begin{itemize}
    \item The correlation is $-0.7233$.
    \item Any answer in the interval $\left[-0.724, -0.720  \right]$ was accepted.
    \item Interpretation (not asked): There is a strong negative linear relationship between the truck's mileage and its advertised price.
  \end{itemize}
  \textbf{Q2:}
  \begin{itemize}
    \item Answer: \emph{There is a negative linear relationship between the age of an F-150 and its price}.
    \item Explanation: The correlation between price and year was positive, but a higher year means a lower age, so there is a negative linear relationship between age and price.
    \item It wasn't necessary to create the $age$ variable, but it helped to check.
    \item Other answer options were either stating a positive relationship between age and price, or no relationship, which we know is wrong!
  \end{itemize}.
\end{frame}
% \end{comment}

\end{document}
